\documentclass{beamer}

\usepackage[brazil]{babel}
\usepackage[T1]{fontenc}
\usepackage[utf8]{inputenc}
\usepackage{amsmath, tikz-qtree}

\usetheme{CambridgeUS}
\usecolortheme{seagull}
\setbeamertemplate{title page}[default][shadow=false]
\setbeamertemplate{headline}{}
\setbeamertemplate{itemize items}[triangle]
\setbeamertemplate{footline}[frame number]{}
\setbeamertemplate{navigation symbols}{}

\title{Esquemas de assinatura digital\\baseados em funções de resumo}
\author{Gustavo Zambonin}
\institute{
  Universidade Federal de Santa Catarina \\
  Departamento de Informática e Estatística \\
  INE5453 - Introdução ao Trabalho de Conclusão de Curso
}
\date{}

\tikzset{every tree node/.style={align=center,anchor=north,font=\tiny}}

\newcommand{\concat}{\, \vert \vert \,}

\begin{document}

\begin{frame}[noframenumbering, plain]
  \titlepage
\end{frame}

\begin{frame}
  \frametitle{Funções de resumo criptográfico}
  \begin{equation*}
    \mathcal{H} : \{0, 1\}^{*} \longrightarrow \{0, 1\}^{n}
  \end{equation*}

  \begin{figure}
    \begin{tikzpicture}
      \node at (-6.5, 0) {entrada $\longrightarrow$};
      \foreach \sgn in {+, -}
        \draw plot[domain=1:5] (-\x, {\sgn 1/20*(3+\x*\x)});
      \foreach \r in {1, 2.3333, ..., 5}
        \draw (-\r, 0) ellipse[x radius=(\r+.5)/20, y radius=1/20*(3+\r*\r)];
      \node at (0.25, 0) {$\longrightarrow$ resumo};
    \end{tikzpicture}
  \end{figure}

  \begin{itemize}
    \item RIPEMD: $n \in \{128, 160, 256, 320\}$
    \item SHA-2, SHA-3, BLAKE: $n \in \{224, 256, 384, 512\}$
    \item Keccak: $n$ arbitrário
  \end{itemize}
\end{frame}

\begin{frame}
  \frametitle{Árvore de Merkle \cite{Merkle1982}}
  \begin{itemize}
    \item $\mathcal{H}(m)$ é uma função de resumo aplicada sobre uma
      mensagem $m$
    \item $\vert \vert$ é o operador de concatenação de palavras
    \item $\mathcal{D}_{n}$ é um bloco de dados qualquer
  \end{itemize}

  \begin{figure}
    \begin{tikzpicture}
      \Tree
        [.\node{$T_{r}$ \\ $\mathcal{H}(T_{0} \concat T_{1})$};
          [.\node{$T_{0}$ \\ $\mathcal{H}(T_{0, 0} \concat T_{0, 1})$};
            [.\node{$T_{0, 0}$ \\ $\mathcal{H}(\mathcal{D}_{0})$};
              [.\node[draw]{$\mathcal{D}_{0}$}; ]
            ]
            [.\node{$T_{0, 1}$ \\ $\mathcal{H}(\mathcal{D}_{1})$};
              [.\node[draw]{$\mathcal{D}_{1}$}; ]
            ]
          ]
          [.\node{$T_{1}$ \\ $\mathcal{H}(T_{1, 0} \concat T_{1, 1})$};
            [.\node{$T_{1, 0}$ \\ $\mathcal{H}(\mathcal{D}_{2})$};
              [.\node[draw]{$\mathcal{D}_{2}$}; ]
            ]
            [.\node{$T_{1, 1}$ \\ $\mathcal{H}(\mathcal{D}_{3})$};
              [.\node[draw]{$\mathcal{D}_{3}$}; ]
            ]
          ]
        ]
    \end{tikzpicture}
  \end{figure}
\end{frame}

\begin{frame}
  \frametitle{Assinatura digital}
  \begin{itemize}
    \item busca prover autenticação, integridade e não-repúdio
    \item baseada em criptografia assimétrica
    \begin{itemize}
      \item chaves pública ($P_{k}$) e privada ($S_{k}$) para uma entidade
    \end{itemize}
    \item consiste de uma tripla de algoritmos \cite{Goldreich2004}
    \begin{itemize}
      \item geração de chaves ($\mathcal{G}$),
          assinatura de mensagens ($\mathcal{S}$), \\
          verificação da assinatura ($\mathcal{V}$)
    \end{itemize}
    \item deve existir uma maneira de ligar o assinante à chave
  \end{itemize}

  \begin{figure}
    \begin{tikzpicture}
      \node (sk) at (0, -1) {$S_{k}$};
      \node (pk) at (4, -1) {$P_{k}$};
      \node (hm) at (-1, 0) {$m$};
      \node[draw] (sig) at (0, 0) {$\mathcal{S}$};
      \node (ds) at (2, 0) {\scriptsize{$m \concat
        \mathcal{S}(\mathcal{H}(m))$}};
      \node[draw, circle] (ger) at (2, -2) {$\mathcal{G}$};
      \node[draw] (ver) at (4, 0) {$\mathcal{V}$};
      \node (res) at (5.5, 0) {\{0, 1\}};
      \draw[-latex] (ger) to (0, -2) to (sk);
      \draw[-latex] (ger) to (4, -2) to (pk);
      \draw[-latex] (sk) -- (sig);
      \draw[-latex] (hm) -- (sig);
      \draw[-latex] (sig) -- (ds);
      \draw[-latex] (ds) -- (ver);
      \draw[-latex] (pk) -- (ver);
      \draw[-latex] (ver) -- (res);
    \end{tikzpicture}
  \end{figure}
\end{frame}

\begin{frame}
  \frametitle{Esquema de assinatura de Lamport--Diffie \cite{Lamport1979}}
  \begin{itemize}
    \item acredita-se ser seguro o suficiente em computadores quânticos
    \item par de chaves deve ser utilizado uma única vez
    \begin{itemize}
      \item cada uso posterior divide o nível de segurança pela
        \href{https://crypto.stackexchange.com/a/2642}{metade}
    \end{itemize}
    \item descrição do algoritmo
    \begin{itemize}
      \item $\mathcal{G}$: $S_{k} = \{y_{i, j} : 1 \leq i \leq n, \;
        j \in \{0, 1\}\}$ gerados aleatoriamente \\
        \hspace{1.3em}$P_{k} = \{\mathcal{H}(y) : y \in S_{k}\}$
    \item $\mathcal{S}$: $\mathcal{H}(m)
        = h_{1}, \dots h_{n} \in \{0, 1\}$ \\
        \hspace{1.3em}$\mathcal{S}(\mathcal{H}(m))
          = (y_{1, h_{1}}, \dots, y_{n, h_{n}})$
      \item $\mathcal{V}$:
          $\forall y_{i, h_{i}} \in \mathcal{S}(\mathcal{H}(m)), \;
            \mathcal{H}(y_{i, h_{i}}) = P_{k_{i, h_{i}}}, 1 \leq i \leq n$
    \end{itemize}
  \end{itemize}
\end{frame}

\begin{frame}
  \frametitle{Esquema de assinatura de Merkle}
  \begin{itemize}
    \item árvore de Merkle + esquema de assinatura única (e.g. Lamport)
    \item assina um número limitado ($t \leq 25$) de mensagens
      $\{m_{1}, \dots, m_{2^{t}}\}$
    \item definição do algoritmo para uma árvore $T$
      \begin{itemize}
        \item $\mathcal{G}$: $\mathbf{S_{k}}
            = \{(S_{k_{i}}, P_{k_{i}}) : 1 \leq i \leq 2^{t}\}$ \\
          \hspace{1.3em}$\mathbf{P_{k}} = \mathcal{H}(T_{r})$,
            com $\{\mathcal{H}(P_{k_{i}}) : 1 \leq i \leq 2^{t}\}$
            como folhas da árvore
        \item $\mathcal{S}$: escolha de um par não utilizado
          $(S_{k_{i}}, P_{k_{i}})$, produz $\mathcal{S}'$ \\
          \hspace{1.3em}$A =$ nodos para chegar à raiz + seus irmãos \\
          \hspace{1.3em}$\mathcal{S}(m_{i})
            = \{\mathcal{S}' \concat P_{k_{i}} \concat A\}$
        \item $\mathcal{V}$: verifica $\mathcal{S}'$ e constrói a sub-árvore
          $T'$; $\mathcal{H}(T'_{r}) = \mathbf{P_{k}}$
      \end{itemize}
  \end{itemize}
\end{frame}

\begin{frame}
  \frametitle{Esquemas mais complexos}
  \begin{itemize}
      \item Winternitz (1989, \cite{Dods2005})
    \begin{itemize}
      \item baseado em aplicações repetidas de $\mathcal{H}$
      \item chave pública muito menor
      \item processo de verificação mais eficiente
    \end{itemize}
    \item CMSS (2005, \cite{Buchmann2006})
    \begin{itemize}
      \item encadeamento de árvores
      \item $2^{40}$ mensagens
      \item guarda apenas a semente do PRNG em $P_{k}$
    \end{itemize}
    \item GMSS (2007, \cite{Buchmann2007})
    \begin{itemize}
      \item \emph{Generalized Merkle Signature Scheme}
      \item $2^{80}$ mensagens
    \end{itemize}
    \item XMSS (2011, \cite{Buchmann2011})
    \begin{itemize}
      \item \emph{eXtended Merkle Signature Scheme}
      \item uma mensagem comprometida não expõe outras (\emph{forward secrecy})
    \end{itemize}
  \end{itemize}
\end{frame}

\begin{frame}[allowframebreaks]
  \frametitle{Referências}
  \bibliography{introduction}
  \bibliographystyle{abbrv}
\end{frame}

\end{document}
