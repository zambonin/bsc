\documentclass[12pt]{report}

\usepackage[brazil]{babel}
\usepackage[T1]{fontenc}
\usepackage[utf8]{inputenc}
\usepackage[a4paper, margin=2.75cm]{geometry}
\usepackage[colorlinks, urlcolor=blue, citecolor=red]{hyperref}
\usepackage{amsmath, amsfonts, enumitem, tikz, parskip}

\newcommand{\hh}{$\mathcal{H}$}
\newcommand{\pk}{$\mathcal{P}_k$}
\newcommand{\sk}{$\mathcal{S}_k$}
\newcommand{\hash}[2][]{\mathcal{H}^{#1}(#2)}
\newcommand{\concat}{\, \vert \vert \,}
\newcommand{\binwds}[1]{\{0, 1\}^{#1}}
\newcommand{\length}[1]{\vert #1 \vert}

\def\precircle{(0.00, 0) circle (1.25cm)}
\def\seccircle{(1.75, 0) circle (1.25cm)}
\def\colcircle{(1.75, 0) circle (0.75cm)}

\colorlet{circle edge}{black!50}
\colorlet{circle area}{black!35}

\tikzset{
  filled/.style={fill=circle area, draw=circle edge, thick},
  outline/.style={draw=circle edge, thick}
}

\title{Otimização de desempenho do esquema de assinatura digital única Winternitz}
\author{Gustavo Zambonin}
\date{}

\begin{document}

\maketitle

\chapter{Introdução}

A aplicação de protocolos criptográficos é essencial no contexto da validação e
proteção de quaisquer comunicações realizadas por um conjunto de entidades,
sejam estas dispositivos eletrônicos ou indivíduos, em virtude da possível
criticalidade e sensibilidade atribuídas aos dados transmitidos. Esquemas de
assinatura digital são comumente utilizados para assegurar este processo de
maneira formal~\cite{Goldreich:2004:FCV:975541}, através da autenticidade e
não-repúdio do remetente e certeza da integridade dos dados, a fim de
traduzir o resguardo provido por uma assinatura de próprio punho no mundo real.

Na prática, a maior parte destes esquemas utilizam como alicerce algorítmico
criptossistemas assimétricos baseados em problemas ``difíceis'' da teoria
dos números, como a fatoração de inteiros ou resolução do logaritmo discreto,
ambos para números grandes. Este fato provê a segurança necessária para os
esquemas em computadores clássicos (eletrônicos), por conta da inexistência de
algoritmos que resolvem estes problemas em tempo polinomial, até o momento.
Entretanto, em computadores quânticos, algoritmos dessa forma já existem -- em
especial, o algoritmo de Shor~\cite{Shor:1997:PAP:264393.264406} -- efetivamente
tornando estes esquemas clássicos inseguros neste novo contexto.

Para combater esta situação, a \emph{criptografia pós-quântica} encarrega-se de buscar
algoritmos criptográficos cuja segurança é considerada suficiente mesmo
utilizando-se de um computador quântico e ataques especializados, como o
algoritmo de Grover~\cite{Grover:1996:FQM:237814.237866}. Esta área conta com
diversas abordagens diferentes: a criptografia baseada em reticulados,
polinômios de múltiplas variáveis sobre um corpo finito, teoria de códigos,
morfismos entre curvas elípticas supersingulares e criptossistemas simétricos.
Entretanto, reduções de segurança formais não existem para alguns destes
métodos, e para outros, o tamanho das chaves impossibilita a utilização destes
em aplicações práticas~\cite{Bernstein2017}.

Não obstante, uma abordagem distinta de esquema de assinatura digital
resistente a computadores quânticos pode ser obtida apenas com funções de
resumo criptográfico, construídas a partir de funções de mão
única~\cite{cryptoeprint:2005:328}. De fato, estas funções, desde que apresentem
requisitos de segurança como resistência à segunda pré-imagem e à colisões, são
necessárias e suficientes para a construção de esquemas bem comportados e
seguros~\cite{Rompel:1990:OFN:100216.100269}. Visto que estas funções são
estudadas exaustivamente por conta de sua vasta presença em diversos âmbitos da
segurança da informação, reduções de segurança são mais comuns em relação a
outras abordagens pós-quânticas, e tamanhos de chaves e assinaturas não são
proibitivos.

Esquemas de assinatura digital baseados em funções de resumo criptográfico
consistem da utilização de um esquema de assinatura digital \emph{única},
onde apenas uma mensagem pode ser assinada de modo seguro,
ou sua combinação com a estrutura de dados chamada de árvore de
Merkle~\cite{Merkle:1989:CDS:118209.118230}, que abriga diversos pares de chave do
esquema supracitado como suas folhas, e reduz a verificação destes para uma
única chave, codificada em sua raiz. Esta árvore é construída com a
concatenação de resumos criptográficos do conteúdo dos nós, habilitando assim a
assinatura de diversas mensagens. Como uma função específica não é necessária,
é possível obter uma grande variedade de esquemas, garantindo a versatilidade
destas abordagens.

Embora os esquemas iniciais tenham sido construídos sem atenção particular à
eficiência de modo geral (e.g. o esquema de assinatura única de Lamport-Diffie
\cite{Lamport1979} assina apenas um \emph{bit} de informação em sua forma mais
simples), muitos resultados práticos demonstram a redução contínua do tempo de
verificação da assinatura, tamanho e tempo para geração do par de chaves e
assinatura, bem como avanços teóricos possibilitam a utilização de funções com
requisitos de segurança mínimos~\cite{cryptoeprint:2017:965},
garantem o conceito de sigilo encaminhado~\cite{Buchmann:2011:XPF:2184003.2184011}
(i.e. comprometimento de uma chave não
implica na segurança de mensagens que utilizaram esta chave anteriormente) e da
ausência de estado~\cite{Bernstein2015} (i.e. esquema não necessita registrar
quais chaves de assinatura única já foram utilizadas).

Neste trabalho, foca-se no esquema de assinatura digital única Winternitz, e
apresenta-se uma customização para o esquema na forma de um parâmetro extra,
que habilita a redução de tempos de verificação de assinatura em troca de
mais computação na geração desta, ou vice-versa. Ademais, as consequências
desta otimização são verificadas em esquemas mais complexos, como os baseados
em árvores de Merkle. Este trabalho é uma versão expandida, e portanto didática,
do artigo a ser publicado como~\cite{Peri1806:Tuning}.

\section{Objetivos}

\begin{itemize}

\item \emph{Objetivo geral.} Apresentar um estudo detalhado sobre o esquema de
    assinatura digital única Winternitz, contextualizando-o junto ao estado da
    arte, observando o refinamento e utilização deste em vários outros esquemas
    a fim de habilitar o gerenciamento de múltiplas assinaturas, fundamentando
    a criação de uma otimização que afeta os tempos de execução da criação e
    verificação de uma assinatura.
    
\item \emph{Objetivos específicos.} Descrever os esquemas de assinatura digital
    única Lamport-Diffie e Winternitz, e sua variante \textsc{Wots+}. Descrever
    os esquemas de assinatura digital baseados em árvores de Merkle: \emph{Merkle
    Signature Scheme}, e as famílias XMSS e SPHINCS. Discutir as consequências
    da modificação do esquema Winternitz no contexto destes, e mensurar desempenho
    onde aplicável.

\end{itemize}

\chapter{Primitivas criptográficas}

Neste capítulo, são mostradas breves explicações sobre os conceitos
necessários a fim de entender inteiramente um esquema de assinatura digital:
a função de resumo criptográfico, utilizada para que o processo de assinatura
seja menos custoso e mais seguro, e um exemplo da construção teórica por trás
deste tipo de função; os conceitos de criptografia simétrica e assimétrica, e
uma simples comparação entre os mesmos, com foco na variante assimétrica,
novamente apresentando um exemplo comum -- mas não resistente a computadores
quânticos -- e, por fim, a definição formal de um esquema de assinatura digital
agregando estas noções.

\section{Funções de resumo criptográfico}

Funções criadas com o intuito de resumir dados, ou seja, reduzir uma mensagem
potencialmente grande para uma palavra pequena e identificável, podem possuir
várias propriedades, apresentadas abaixo de acordo com~\cite{Menezes:1996:HAC:548089}.
Tome uma função $\mathcal{H} : X \longrightarrow Y$. Comumente, os elementos da imagem
de \hh{} são chamados de resumos. É importante notar que um problema é considerado `'difícil`', ou
computacionalmente impraticável, quando o tempo ou recursos gastos para esta
computação excedem a validade ou utilidade da informação desejada.

\begin{enumerate}[label=(\roman*)]
    \item O cálculo de todo resumo deve ser \emph{computacionalmente fácil};
    
    \item \hh{} pode apresentar \emph{compressão},
        ou seja, $\forall x \in X, y \in Y, \length{x} > \length{y}$;
    
    \item \hh{} pode apresentar resistência à pré-imagem (\textsc{Pre}),
        caracterizada pelo seguinte comportamento: fornecido um resumo
        $h \in Y$, é computacionalmente inviável achar a mensagem
        original $m \in X$ que gerou $h$ através de $\hash{m} = h$;

    \item \hh{} pode apresentar resistência à segunda pré-imagem (\textsc{Sec}),
        caracterizada pelo seguinte comportamento: fornecida uma
        mensagem $m_0 \in X$, é computacionalmente inviável achar
        uma mensagem $m_1 \in X$ tal que $m_0 \neq m_1$ e
        $\hash{m_0} = \hash{m_1}$;

    \item Para quaisquer duas mensagens $m_0, \; m_1 \in X$ e $m_0 \neq m_1$,
    $\hash{m_0} = \hash{m_1}$. \hh{} é considerada resistente à colisões
    (\textsc{Col}) se isto não pode ser resolvido de maneira eficiente.
\end{enumerate}





Uma função de resumo \hh{} mapeia valores deterministicamente entre dois
conjuntos, a fim de prover fácil identificação e busca de dados.




O domínio pode ter tamanho infinito, e neste caso a função pode ser
chamada de função de compressão; a imagem deve ser estritamente menor do que o
domínio e finita, e elementos deste conjunto são chamados de resumos. É
desejável para \hh{} que estes mapeamentos ocorram de tal maneira que não
ocorra uma relação aparente entre entradas e saídas da função. Funções de
resumo adicionadas de propriedades que tornam-as adequadas para utilização no
contexto de segurança da informação são chamadas de funções de resumo
criptográfico, e possibilitam a certeza da integridade de dados, mesmo que
armazenados em um dispositivo inseguro.

Tome $X : \binwds{*}$ e $Y : \binwds{n}$, $n \in \mathbb{N}$. Então,
$\mathcal{H} : X \longrightarrow Y$. De acordo com
\cite{stinson2005cryptography}, para que qualquer \hh{} seja considerada
criptográfica, deve ser difícil resolver os três problemas listados abaixo.


Note que \textsc{Sec} e \textsc{Col}. apresentam uma sutil diferença: no
primeiro, um adversário não pode escolher $m_0$, enquanto no segundo, quaisquer
pares de mensagens podem ser testados. A resistência à colisão, portanto,
implica na resistência à segunda pré-imagem, visto que basta um adversário
fixar $m_0$ para simular o cômputo de $m_1$. Outra característica desejada é o
efeito avalanche, baseado no conceito de difusão
\cite{Stallings:2010:CNS:1824151}: trocar apenas um \emph{bit} da mensagem $m$
deve modificar cerca de metade dos \emph{bits} do resumo, e vice-versa.

\begin{figure}[h]
  \centering
  \begin{tikzpicture}
    \begin{scope}[fill opacity=0.5]
      \clip \precircle;
      \fill[filled] \seccircle;
      \fill[filled] \colcircle;
    \end{scope}
    \draw[outline] \precircle node {\textsc{Pre}};
    \draw[outline] \seccircle node {};
    \draw node[right=1.75cm, above=0.75cm] {\textsc{Sec}};
    \draw[outline] \colcircle node {\textsc{Col}};
  \end{tikzpicture}
  \caption{Diagrama de Venn das resistências de uma função de resumo
    criptográfico.}
  \label{fig:1}
\end{figure}

Na Figura \ref{fig:1}, estão destacados os requisitos comuns para a utilização
de funções de resumo criptográfico no contexto de esquemas de assinatura
digital, em vista da possibilidade de uma entidade maliciosa, geralmente
chamada de adversário, desejar produzir assinaturas forjadas. É possível
constatar que, embora exista uma divisão estrita entre \textsc{Pre} e
\textsc{Sec}, observa-se que na prática, é possível assumir que a segunda
implica a primeira resistência \cite{Menezes:1996:HAC:548089}.

Enumeram-se algumas aplicações comuns para estas funções: a verificação da
integridade de um arquivo, i.e. determinar se mudanças neste foram feitas ao
longo de uma transmissão, ou qualquer outro evento; a fim de evitar o
armazenamento de senhas em texto plano, é possível manter apenas o resumo
criptográfico destas, e no momento da autenticação do usuário perante o
serviço, comparar apenas estes resumos\footnote{É possível armazenar tabelas de
resumos pré-computados a fim de atacar serviços que não empregam uma maneira
mais elaborada de autenticação (por exemplo, um valor aleatório
concatenado ao resumo criptográfico da senha do usuário).}; resumos
criptográficos são comumente empregados como identificadores únicos para um
arquivo (e.g. \emph{commits} em um sistema de controle de versões); entre
outras aplicações, como a geração de números pseudoaleatórios.

\subsection{Construção esponja}

A construção esponja \cite{SpongeReference}, de característica iterativa,
permite a generalização de funções de resumo, naturalmente com saídas de
tamanho fixo, para funções com saídas de tamanho arbitrário, baseadas em uma
função interna, geralmente uma permutação $f$ de tamanho fixo $b$. Este valor,
também chamado de largura, é composto da adição da taxa de \emph{bits} $r$ e da
capacidade $c$. Assim, a construção opera em um estado de $b = r + c$
\emph{bits}.

O estado inicial, análogo a um vetor de inicialização no contexto de algoritmos
criptográficos, não necessita de valores especiais e é inicializado com valores
nulos. A entrada $m$ é preenchida com uma função de preenchimento \texttt{pad}
de tal modo que $r \mid \length{m}$, e dividida em blocos de tamanho $r$. A
fase de absorção de $m$ pela esponja procede da seguinte maneira: a operação de
ou exclusivo ($\oplus$) é calculada entre os blocos e os estados da construção,
intercalados por aplicações de $f$.

\begin{figure}[ht]
    \centering
    
    \begin{tikzpicture}[scale=0.45]

\tikzset{SpongePerm/.style=rounded corners=4pt,};
\tikzset{edge/.style=->};
\tikzset{edgee/.style=<->};
\tikzset{
    XOR/.style={
        draw,circle,append after command={
            [shorten >=\pgflinewidth, shorten <=\pgflinewidth,]
            (\tikzlastnode.north) edge (\tikzlastnode.south)
            (\tikzlastnode.east) edge (\tikzlastnode.west)
        }
    }
}

\begin{scope}[xshift=0cm]
    \draw[thick] (0,0) rectangle ++(1,10);
    \draw[thick] (0,3) -- ++(1,0);
    \node[XOR,thick] (xm0) at (1+1.5,8) {};
    \draw[edge,thick] (1,8) -- (xm0);
    \draw[edge,thick] (1,2) -- ++(3,0);
    \draw[edge,thick] (1+1.5,10.5) node[above] {\large $m_{0}$} -- (xm0);  
    \draw[edge,thick] (xm0) -- ++(1.5,0);
    \draw[edgee,anchor=east] (-1,3) -- node[left] {$r$ bits} ++(0,7);
    \draw[edgee,anchor=east] (-1,0) -- node[left] {$c$ bits} ++(0,3);
\end{scope}
  
\begin{scope}[xshift=4cm]
    \draw[SpongePerm] (0,0) rectangle node {\large$f$} ++(1,10);
    \node[thick] (xm1) at (1+1.5,8) {$\dots$};
    \draw[edge,thick] (1,8) -- (xm1);
    \draw[edge,thick] (xm1) -- ++(1.5,0);
    \node[thick] (xm1) at (1+1.5,2) {$\dots$};
    \draw[edge,thick] (1,2) -- (xm1);
    \draw[edge,thick] (xm1) -- ++(1.5,0);
\end{scope}

\begin{scope}[xshift=8cm]
    \draw[SpongePerm] (0,0) rectangle node {\large$f$} ++(1,10);
    \node[XOR,thick] (xm1) at (1+1.5,8) {};
    \draw[edge,thick] (1,8) -- (xm1);
    \draw[edge,thick] (1,2) -- ++(3,0);  
    \draw[edge,thick] (1+1.5,10.5) node[above] {\large $m_{i}$} -- (xm1);
    \draw[edge,thick] (xm1) -- ++(1.5,0);
\end{scope}
  
\begin{scope}[xshift=12cm]
    \draw[SpongePerm] (0,0) rectangle node {\large$f$} ++(1,10);
    \draw[edge,thick] (1,2) -- ++(3,0);
    \draw[edge,thick] (1,8) -- ++(3,0);  
\end{scope}

\begin{scope}[xshift=16cm]
    \draw[thick] (0,0) rectangle ++(1,10);
    \draw[thick] (0,3) -- ++(1,0);
    \draw[edge,thick] (1,2) -- ++(3,0);
    \draw[edge,thick] (1,8) -- ++(3,0);  
    \draw[edge,thick] (1+1.5,8) -- ++(0,2.5) node[above] {\large $z_{0}$};
    \draw[dashed,thick] (-1.5,-1.5) -- ++(0,13);
\end{scope}

\begin{scope}[xshift=20cm]
    \draw[SpongePerm] (0,0) rectangle node {\large$f$} ++(1,10);
    \node[thick] (xm1) at (1+1.5,8) {$\dots$};
    \draw[edge,thick] (1,8) -- (xm1);
    \draw[edge,thick] (xm1) -- ++(1.5,0);
    \node[thick] (xm1) at (1+1.5,2) {$\dots$};
    \draw[edge,thick] (1,2) -- (xm1);
    \draw[edge,thick] (xm1) -- ++(1.5,0);
\end{scope}

\begin{scope}[xshift=24cm]
    \draw[SpongePerm] (0,0) rectangle node {\large$f$} ++(1,10);
    \draw[edge,thick] (1,8) -- ++(1.5,0) -- ++(0,2.5) node[above] {\large $z_{j}$};
\end{scope}

\end{tikzpicture}
    
    \caption{Descrição gráfica da construção esponja, adaptado de~\cite{TikZ:for:Cryptographers}.}
    \label{fig:my_label}
\end{figure}


Ao término do processamento dos blocos, a fase de compressão é iniciada, onde
$n$ blocos de tamanho $r$ compõem a saída da função, novamente intercalados por
aplicações de $f$, onde $n$ é parametrizável pelo usuário. Os últimos $c$ bits
do estado nunca são diretamente afetados pelos blocos, e também nunca revelados
durante a fase de compressão. Essencialmente, estão correlacionados com o nível
de segurança da construção esponja. Assim, uma função esponja pode ser definida
como $\textsc{Sponge}[f, \texttt{pad}, r]$.

A função esponja \textsc{Keccak} \cite{KeccakReference} é definida a partir
desta construção, e pode agir como uma função de resumo criptográfico. Existem
sete permutações passíveis de utilização nesta função: defina $w = 2^{\ell}, \;
\ell \in \{0, \dots, 6\}$.  Estas são chamadas de $\textsc{Keccak}-f[b]$, onde
$b = 25w$, cujo estado $a$ é descrito como uma estrutura tridimensional com
elementos em $\mathbb{F}_2$, de dimensões $5 \times 5 \times w$. Esta
permutação é iterativa e consiste de um número de rodadas $n_R$. Cada rodada
$R$, por sua vez, consiste da composição de cinco etapas: $R = \iota \circ \chi
\circ \pi \circ \rho \circ \theta$.

\begin{enumerate}

  \item[Etapa $\theta$:] Calcula o ou exclusivo entre um elemento de $a$ e
    todos os elementos das colunas adjacentes a este.

  \item[Etapa $\rho$:] Dispersa os elementos entre cortes transversais
    verticais de $a$.

  \item[Etapa $\pi$:] Rearranja elementos em cortes transversais horizontais de
    $a$.

  \item[Etapa $\chi$:] Modifica uma elemento de uma linha de $a$ de acordo com
    uma função não-linear de dois outros bits adjacentes. Análogo a uma caixa-S.

  \item[Etapa $\iota$:] Calcula o ou exclusivo entre o estado $a$ e uma
    sequência gerada por um \emph{linear-feedback shift register} alimentado
    pelo índice da rodada atual, tornando a rodada assimétrica.

\end{enumerate}

Tome \texttt{pad10*1} como uma função que gera palavras que iniciam e terminam
com $1$, e têm número não-negativo de zeros. Formalmente, para uma mensagem
qualquer $m$ e um tamanho de saída $d \in \mathbb{N}^{*}$, a função esponja é
definida como
\begin{equation}
  \textsc{Keccak}[r, c](m, d)
    = \textsc{Sponge}[\textsc{Keccak}-f[r + c], \texttt{pad10*1}, r]
\end{equation}

onde $r$ tem um valor padrão de $1600 - c$. Assim,
\begin{equation}
  \textsc{Keccak}[c] = \textsc{Keccak}[1600 - c, c].
\end{equation}

Finalmente, as funções padronizadas em \cite{Dworkin2015} como a família SHA-3
são definições de \textsc{Keccak} com parâmetros fixos, e.g.
\begin{equation}
  \text{SHA3-}256(m) = \textsc{Keccak}[512](m \; \vert \vert \; 01, 256).
\end{equation}

\section{Criptografia simétrica}

Algoritmos criptográficos que utilizam a mesma chave para criptografar o
texto plano e descriptografar o texto correspondente cifrado são classificados
como algoritmos de criptografia simétrica. A chave representa um segredo
compartilhado entre entidades em uma comunicação segura. Porém, a necessidade
de um canal seguro para o estabelecimento desta chave apresenta-se como uma
desvantagem deste tipo de criptografia. Geralmente, cifras de bloco (DES, AES)
ou de fluxo (RC4, Salsa20) são a base para estes algoritmos. Utilizando estas
como alicerce, é possível construir funções de resumo criptográfico: por
exemplo, a construção Merkle-Damgård, base para as funções MD5, SHA1 e SHA2,
utiliza uma função de compressão única, obtida a partir de uma cifra de bloco.

\section{Criptografia assimétrica}

Em contrapartida, a criptografia assimétrica, ou criptografia de chaves
públicas, engloba os algoritmos que utilizam um par de chaves: a chave privada
(\sk{}), conhecida apenas pela entidade que a gerou, e a chave pública (\pk{}),
distribuída livremente. Isto possibilita o uso livre de \pk{} para a
comunicação segura com o detentor da chave sem a necessidade de um canal
seguro, em virtude da construção dos algoritmos. A segurança destes depende da
``dificuldade'' computacional de determinar uma chave privada a partir da chave
pública, e também do armazenamento de \sk{} em um lugar seguro. Problemas em
teoria de números e álgebra que atualmente não admitem soluções em tempo
polinomial são comumente utilizados como base para algoritmos assimétricos.
Porém, percebe-se que, com a introdução de um computador quântico, estes
problemas podem ser resolvidos de maneira significativamente mais rápida, como
visto em \cite{Shor:1997:PAP:264393.264406}.

\subsection{O criptossistema RSA}

O algoritmo conhecido como RSA \cite{Rivest:1978:MOD:359340.359342} é uma
implementação de criptografia assimétrica amplamente utilizada. É baseado na
dificuldade de fatorar o produto de dois números primos suficientemente
grandes\footnote{O algoritmo é baseado no problema RSA, definido como realizar
uma operação de chave privada no algoritmo RSA utilizando apenas \pk{}.
Acredita-se que este problema seja equivalente à fatoração de inteiros
\cite[3.30]{Menezes:1996:HAC:548089}.}. Em virtude de seu baixo desempenho
computacional, geralmente apenas um resumo criptográfico da mensagem desejada é
codificado por este algoritmo. Abaixo, uma descrição do funcionamento do
algoritmo. Tome $\phi(x)$ como a função totiente de Euler, que representa a
quantidade de números relativamente primos a $x$.

\begin{enumerate}

  \item[] \emph{Geração de chaves.} Gere dois números primos $p, q$
    aleatoriamente, suficientemente grandes e de tamanhos similares. Compute
    $n = p q$ e $\phi(n) = (p - 1) (q - 1)$. Selecione um número aleatório $e$
    relativamente primo a $\phi(n)$. Então, use o algoritmo de Euclides
    estendido para computar $d$ tal que $ed \equiv 1 \pmod{\phi(n)}$), i.e.
    a inversa multiplicativa modular de $e$. Finalmente, $\mathcal{S}_k = d$ e
    $\mathcal{P}_k = (n, e)$.

  \item[] \emph{Codificação.} Obtém-se \pk{} da entidade para qual
    deseja-se criptografar uma mensagem. Transforma-se uma mensagem $m$ em um
    inteiro no intervalo $[0, n - 1]$ através de uma função de preenchimento.
    O texto cifrado $c = m^e \pmod{n}$ é calculado através de um algoritmo
    como a exponenciação quadrática e enviado para a entidade desejada.

  \item[] \emph{Decodificação.} O receptor da mensagem calcula $m = c^d
    \pmod{n}$.

  \item[] \emph{Demonstração}. Para demonstrar $m^{ed} \equiv m \pmod{n}$, é
    suficiente mostrar $m^{ed} \equiv m \pmod{p}$ e $m^{ed} \equiv m \pmod{q}$,
    pelo Teorema Chinês do Resto. Se $m \equiv 0 \pmod{p}$, então
    $gcd(m, p) = p$ e certamente $m^{ed} \equiv 0 \equiv m \pmod{n}$.
    Se $m \not\equiv 0 \pmod{p}$, então $mdc(m, p) = 1$ e pelo Pequeno Teorema
    de Fermat, $m^{p - 1} \equiv 1 \pmod{p}$. Reescrevendo o produto $ed$ como
    $ed = 1 + y\phi(n) = 1 + y(p - 1)(q - 1), \; y \in \mathbb{N}$, então
    \begin{equation}
      m^{ed} \equiv m^{1 + y(p-1)(q-1)} \equiv (m^{p-1})^{y(q-1)}m
        \equiv 1^{y(q-1)}m \equiv m \pmod{p}.
    \end{equation}
    Analogamente, substituindo $p$ por $q$ no argumento acima, tem-se a prova
    que $\forall m \in \mathbb{N}, \; m^{ed} \equiv m \pmod{n}$.

\end{enumerate}

\section{Esquemas de assinatura digital}

Um esquema de assinatura digital é uma construção matemática que habilita a
demonstração de certas propriedades sobre mensagens assinadas: nomeadamente,
a autenticação do remetente, onde esta entidade pode ser facilmente
identificada como a emissora da assinatura digital; a integridade da mensagem,
i.e. a certeza de que esta não foi modificada ao ser transmitida por um canal
possivelmente inseguro; e o não-repúdio do remetente, onde não é possível negar
que uma mensagem foi assinada e enviada, após este fato.

\begin{figure}[ht]
  \centering
  \begin{tikzpicture}
    \node (hm) at (-1.25, 0) {$m$};
    \node (in) at (0, -2) {$1^n$};
    \node (sk) at (0, -1) {\sk{}};
    \node (pk) at (4, -1) {\pk{}};
    \node (ds) at (2, 0)
      {\scriptsize{$m \concat \textsc{Sig}(\mathcal{S}_k, \hash{m})$}};
    \node (res) at (5.5, 0) {\scriptsize\{0, 1\}};
    \node[draw] (sig) at (0, 0) {\textsc{Sig}};
    \node[draw] (gen) at (2, -2) {\textsc{Gen}};
    \node[draw] (ver) at (4, 0) {\textsc{Ver}};
    \draw[-latex] (gen) to (1.25, -1) to (sk);
    \draw[-latex] (gen) to (2.75, -1) to (pk);
    \draw[-latex] (sk) -- (sig);
    \draw[-latex] (hm) -- (sig);
    \draw[-latex] (sig) -- (ds);
    \draw[-latex] (ds) -- (ver);
    \draw[-latex] (pk) -- (ver);
    \draw[-latex] (ver) -- (res);
    \draw[-latex] (in) -- (gen);
  \end{tikzpicture}
  \caption{Funcionamento típico de um esquema de assinatura digital.}
  \label{fig:2}
\end{figure}

Esquemas de assinatura digital são fortemente baseados em criptografia de
chaves públicas, e consistem de três algoritmos: a geração de chaves
$\textsc{Gen}(1^n)$, que gera um par de chaves aleatório $(\mathcal{P}_k,
\mathcal{S}_k)$ com parâmetro de segurança $n$; o algoritmo de assinatura
$\textsc{Sig}(\mathcal{S}_k, m)$, que produz uma assinatura $\sigma$ para uma
mensagem $m$; e o algoritmo de verificação $\textsc{Ver}(\mathcal{P}_k, m,
\sigma)$, que retorna o estado de validade da assinatura como um valor verdade
binário. De acordo com \cite{Goldreich2004}, todas as assinaturas geradas por
\textsc{Sig} devem ser verificáveis por \textsc{Ver} utilizando todas as chaves
geradas por \textsc{Gen}. Formalmente, $\forall (p, s) \in
\textsc{Gen}^{\rightarrow}(1^n)$ e $\forall w \in \{0, 1\}^{*}$,
\begin{equation}
    \text{Pr}[\textsc{Ver}(p, w, \textsc{Sig}(s, w)) = 1] = 1.
\end{equation}

Na Figura \ref{fig:2}, é possível visualizar um diagrama do comportamento de
um esquema de assinatura digital genérico. Note que $\sigma$ geralmente é
composto da concatenação da mensagem original com a assinatura do resumo
criptográfico desta, embora a saída do algoritmo \textsc{Sig} consista apenas
da aplicação de uma função interna a este ao resumo.

\bibliographystyle{alpha}
\bibliography{ref}

\end{document}
