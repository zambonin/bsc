\newcommand{\rfc}{Prof. Custódio}
\newcommand{\dan}{Prof. Daniel Panario}
\newcommand{\hh}{$\mathcal{H}$}
\newcommand{\hash}[2][]{\mathcal{H}^{#1}(#2)}

\section*{Past meetings}

\emph{29th of April, 2017.} Definition of the subject for the initial research
and redaction of the monograph, as well as discussion about various related
topics on post-quantum cryptography. Suggestions for the scope of the research
at an undergraduate level. Possibility of future works given the popularity of
the subject. It was asked by \rfc{} that a presentation containing details
about the simpler schemes (Lamport-Diffie -- LD-OTS -- and Merkle) should be
produced and given to himself and \dan{}.

\emph{10th of May, 2017.} The presentation asked on the previous meeting was
given, and it generated quite a few moments of discussion, specifically about
the motivation behind Lamport-Diffie by \rfc{}. Other topics were also raised,
such as its key pair size and newer schemes that do not make use of LD-OTS.
\dan{} suggested that a pattern should be identified with regards to common
optimisations and progress of newer schemes, i.e., what the researchers are
heading towards. Further bibliography should be read to answer these questions
and to ascertain the state of the art in hash-based schemes. No new
presentations are needed, for the next meeting is only a few days away.

\emph{15th of May, 2017.} An overview about recent bibliography covering newer
schemes was given and it was asserted that security requirements, key pair and
signature size, and key pair generation and signing time are being reduced. It
is also important to consider whether a scheme is stateful or stateless, i.e.
if it stores information after each signature. \dan{} suggested that high-level
implementations for a handful of schemes should be studied and perhaps
practised -- programming a scheme should help in understanding it and the code
could be part of the final project. \rfc{} noted that it would be interesting
to check if hash-based schemes are being standardised and asked to check drafts
of RFCs.  Motivation of LD-OTS revolves around the dissociation between the
message and the signature, that is composed only of random numbers' hashes
chosen according to each bit of the desired message's hash. In lieu of its
complexity, the proof of security for LD-OTS was demoted to low priority.
Moving on to slightly harder schemes, a new presentation was asked by \rfc{}
about the Winternitz scheme (WOTS) and a variant of Merkle's, eXtended Merkle
Signature Scheme (XMSS).

\emph{29th of May, 2017.} Last meeting, for now, with \dan{} \emph{in loco}. A
previous meeting scheduled for the 22nd was cancelled due to undergraduate
issues. Drafts of RFCs written by known authors were read, they present slight
variations of schemes already described in various papers by these people.
This shows that there is at least some push for standardisation, although
perhaps without immediate meaning. Some bibliography was removed from the pool
since it was only making things more confusing. It was discovered that
high-level implementations are very sparse and generally unable to be used out
of the box, hence it was decided that at least some code would be produced
alongside the monograph. WOTS was presented and it went quite smoothly,
generating lots of discussion and some tasks, such as uncovering more
information about the trade-off parameter and the check-sum used in the scheme,
presenting it with figures to help the reader's understanding, and if its
performance is affected by the hash function used. These shall be written
directly in the monograph. There was no time to discuss XMSS.

\section*{Hash functions}

A hash function \hh{} maps values deterministically from a set with possibly
infinite size to another, strictly smaller set, with a fixed number of
elements. It is desirable for \hh{} that these mappings happen in such a way
that outputs are distinguished evenly, with no apparent relation between them.
Hash functions with added properties that make them suitable for use in
security applications are called \emph{cryptographic} hash functions;
these can provide assurance of data integrity, even if the data is stored
in an insecure place. Outputs of these functions are generally called
message digests, or simply \emph{digests}. According to
\cite{stinson2005cryptography}, for any \hh{} to be considered
cryptographic, it should be difficult, i.e. computationally infeasible,
to solve the three problems listed below:

\begin{enumerate}[label=\roman*.]
    \item Given a digest $h$, find the original
    message $m$ that generated $h$ through $\hash{m} = h$.
    \hh{} is said to be \emph{preimage} resistant if this cannot be solved efficiently.
    \item Given a message $m_0$, find a message
    $m_1$ such that $m_0 \neq m_1$ and $\hash{m_0} = \hash{m_1}$.
    \hh{} is said to be \emph{second preimage} resistant if this cannot
    be solved efficiently.
    \item Given two distinct messages $m_0$ and $m_1$, then $\hash{m_0} = \hash{m_1}$.
    \hh{} is said to be \emph{collision} resistant if this cannot be
    solved efficiently.
\end{enumerate}

Note that ii. and iii. have a slight difference: in the former, an adversary
may not choose the first message, whereas in the latter, it is free to choose
any pair of messages. Another characteristic desired is the avalanche effect,
based on the concept of diffusion (\cite{Stallings:2010:CNS:1824151}), stating
that `'flipping`' a single bit in the message $m$ should change about half of
its digest bits, and vice-versa.