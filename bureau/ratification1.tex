\documentclass[11pt]{letter}

% chktex-file 44

\usepackage[brazil]{babel}
\usepackage[T1]{fontenc}
\usepackage[a4paper, margin=1.5cm]{geometry}
\usepackage[colorlinks, urlcolor=blue, citecolor=red]{hyperref}
\usepackage[utf8]{inputenc}

\usepackage[table]{xcolor}
\usepackage{mathptmx, multirow, hhline}

\begin{document}

\pagestyle{empty}

\begin{center}
  \textbf{
    DEPARTAMENTO DE INFORMÁTICA E ESTATÍSTICA --- CTC --- UFSC \\
    RATIFICAÇÃO DE PLANO DE TRABALHO DO SEMESTRE PARA DESENVOLVIMENTO DE TCC
  }
\end{center}

\vspace{1em}
\setlength\extrarowheight{5pt}
\begin{tabular}{l l}
  \textbf{Disciplina:}  & INE5433 --- Trabalho de Conclusão de Curso I  \\
  \textbf{Curso:}       & Ciência da Computação                         \\
  \textbf{Autor:}       & Gustavo Zambonin                              \\
  \textbf{Título:}      & Esquemas de assinatura digital baseados
                          em funções de resumo criptográficas           \\
  \textbf{Professor responsável:} & Ricardo Felipe Custódio             \\
\end{tabular}


\vspace{1em}
{\large \textbf{Objetivos}}

\noindent \emph{Objetivo geral.} Apresentar um estudo detalhado sobre esquemas
de assinatura digital baseados em funções de resumo criptográficas, partindo de
esquemas de assinatura única, observando o refinamento destes, até o estado da
arte, onde não é necessário saber quantas assinaturas foram geradas
anteriormente, bem como implementações em linguagem de alto nível para a fácil
compreensão destes esquemas.

\noindent \emph{Objetivos específicos.} Descrever os esquemas de assinatura
digital única Lamport-Diffie e Winternitz; descrever os esquemas de assinatura
digital baseado em árvores de Merkle --- \emph{Merkle Signature Scheme},
\emph{eXtended Merkle Signature Scheme}; implementar os esquemas supracitados;
comparar o desempenho destes algoritmos entre si, utilizando funções de resumo
criptográficas e parâmetros internos aos algoritmos distintos, onde aplicável.

\vspace{1em}
{\large \textbf{Cronograma}}

\begin{center}
  \begin{tabular}{|p{4.04cm}|*{12}{c|}}
    \hline & \multicolumn{6}{c|}{\textbf{2017}}
      & \multicolumn{6}{c|}{\textbf{2018}} \\ \cline{2-13}
    \multicolumn{1}{|c|}{\multirow{-2}{*}{\textbf{Etapas}}}
      & \textbf{jul.} & \textbf{ago.} & \textbf{set.}
      & \textbf{out.} & \textbf{nov.} & \textbf{dez.}
      & \textbf{jan.} & \textbf{fev.} & \textbf{mar.}
      & \textbf{abr.} & \textbf{mai.} & \textbf{jun.} \\
    \hline Fundamentação teórica
      & \cellcolor{lightgray} & & & & & & & & & & & \\
    \hline Revisão do estado da arte
      & \cellcolor{lightgray} & \cellcolor{lightgray} & & & & & & & & & & \\
      \hline Desenvolvimento do TCC
      & & \cellcolor{lightgray} & \cellcolor{lightgray}
      & \cellcolor{lightgray} & & & & & & & & \\
    \hline Implementação
      & & & & \cellcolor{lightgray} & \cellcolor{lightgray}
      & \cellcolor{lightgray} & & & & & & \\
    \hline Relatório de TCC I
      & & & & & \cellcolor{lightgray} & & & & & & & \\
    \hline Ajustes na implementação
      & & & & & & & \cellcolor{lightgray} & & & & & \\
    \hline Redação da monografia
      & & & & & & & \cellcolor{lightgray} & \cellcolor{lightgray}
      & \cellcolor{lightgray} & & & \\
    \hline Ajustes na monografia
      & & & & & & & & & \cellcolor{lightgray} & \cellcolor{lightgray} & & \\
    \hline Relatório de TCC II
      & & & & & & & & & & & \cellcolor{lightgray} & \\
    \hline Defesa pública
      & & & & & & & & & & & & \cellcolor{lightgray} \\
    \hline Ajustes finais do TCC
      & & & & & & & & & & & & \cellcolor{lightgray} \\
    \hline
  \end{tabular}

  \vspace*{\fill}
  \fbox{
    \begin{minipage}[c][6em][c]{0.7\textwidth}{
      \center\textbf{
        Preenchimento pelo professor responsável pelo TCC} \\
      }
      \vspace{3mm}
      \qquad $(\quad)$ \ Ciente e de acordo.
      \qquad Assinatura:

      \qquad Data: \_\_ / \_\_ / \_\_\_\_
    \end{minipage}
  }
\end{center}

\end{document}
