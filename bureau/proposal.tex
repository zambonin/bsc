\documentclass{ufsctex/ufsctex}

\usepackage[table]{xcolor}
\usepackage{mdframed, enumitem, multirow, hhline, amssymb}

\newcolumntype{P}[1]{>{\centering\arraybackslash}m{#1}}

\newcommand{\thatcell}[3]{
  \multicolumn{#1}{#2}{\cellcolor{lightgray} \textbf{#3}}
}

\begin{document}

\instituicao[a]{Universidade Federal de Santa Catarina}
\departamento[o]{Departamento de Informática e Estatística}
\curso[o]{Programa de Graduação em Ciência da Computação}
\documento[a]{Monografia}
\titulo{Esquemas de assinatura digital baseados
  em funções de resumo criptográficas}
\autor{Gustavo Zambonin}
\grau{Bacharel em Ciência da Computação}
\local{Florianópolis}
\data{12}{junho}{2017}
\orientador[Orientador]{Prof. Dr. Ricardo Felipe Custódio}
\coorientador[Coorientador]{Prof. Dr. Daniel Panario}
\coordenador[Coordenador]{Prof. Dr. Rafael Luiz Cancian}

\textoResumo{
  Algoritmos utilizados em esquemas de assinatura digital atualmente, como RSA
  e ECDSA, têm sua segurança baseada em calcular a fatoração de números muito
  grandes ou logaritmos discretos. Este tipo de cômputo pode ser realizado por
  um computador quântico suficientemente poderoso, utilizando algoritmos já
  conhecidos (e.g. algoritmo de Shor). Deste modo, para manter o ecossistema de
  assinaturas digitais continuamente seguro, é necessário oferecer alternativas
  pós-quânticas, ou seja, resistentes a computadores quânticos. Este trabalho
  busca apresentar esquemas baseados apenas em funções de resumo
  criptográficas, cuja segurança baseia-se apenas na resistência à colisão da
  função escolhida, com o objetivo de mostrar que a construção de esquemas de
  assinatura seguros independe de problemas considerados difíceis em teoria de
  números ou álgebra, levando em conta apenas algoritmos quânticos, como o
  algoritmo de Grover. Ademais, apresentam-se soluções para alguns problemas
  deste tipo de esquema, como o tamanho e possibilidade de reutilização das
  chaves pública e privada, assim como uma variada gama de algoritmos com estas
  características, particularmente os baseados em árvores de Merkle e variantes
  do esquema Winternitz.
}
\palavrasChave{criptografia, Merkle, assinatura digital, pós-quântico}

\capa
\folhaderosto

\clearpage

\begin{mdframed}[backgroundcolor=lightgray, linewidth=0pt]
    \centering
    \textbf{FOLHA DE APROVAÇÃO DE PROPOSTA DE TCC}
\end{mdframed}

\vspace{-5mm}
\begin{table}[h]
  \begin{tabular}{|>{\bfseries}l|p{6.57cm}|}
    \hline
    Acadêmico(s)            & Gustavo Zambonin                      \\ \hline
    Título do trabalho      & Esquemas de assinatura digital baseados
                              em funções de resumo criptográficas   \\ \hline
    Curso                   & Ciência da Computação/INE/UFSC        \\ \hline
    Área de Concentração    & Matemática da Computação              \\ \hline
  \end{tabular}
\end{table}

\vspace{-2mm}
{\footnotesize\noindent\textbf{
  Instruções para preenchimento pelo \underline{ORIENTADOR DO TRABALHO}:} \\
  \begin{itemize}[leftmargin=3.6mm,label=-]
    \vspace{-4mm}
    \item Para cada critério avaliado, assinale um X na coluna SIM apenas
        se considerado aprovado. Caso contrário, indique as alterações
        necessárias na coluna Observação.
  \end{itemize}
}

\vspace{-3mm}
\begin{table}[hbpt]
  \begin{tabular}{|>{\tiny}m{4.3cm}*{4}{|>{\columncolor{lightgray}\tiny}c}|c|}
    \hline
    \rowcolor{lightgray} & \multicolumn{4}{c|}{\textbf{Aprovado}} & \\
    \hhline{|>{\arrayrulecolor{lightgray}}->{\arrayrulecolor{black}}|
        |---->{\arrayrulecolor{lightgray}}->{\arrayrulecolor{black}}|}
    \rowcolor{lightgray}
    \multicolumn{1}{|c|}{\multirow{-2}{*}{\normalsize\textbf{Critérios}}}
      & \textbf{Sim} & \textbf{Parcial} & \textbf{Não} & \textbf{Não se aplica}
      & \multirow{-2}{*}{\textbf{Observação}} \\ \hline
    1. O trabalho é adequado para um TCC no CCO/SIN
      (relevância / abrangência)?                         & & & & & \\ \hline
    2. O título do trabalho é adequado?                   & & & & & \\ \hline
    3. O tema de pesquisa está claramente descrito?       & & & & & \\ \hline
    4. O problema/hipóteses de pesquisa do trabalho
      está claramente identificado?                       & & & & & \\ \hline
    5. A relevância da pesquisa é justificada?            & & & & & \\ \hline
    6. Os objetivos descrevem completa e claramente
      o que se pretende alcançar neste trabalho?          & & & & & \\ \hline
    7. É definido o método a ser adotado no trabalho?
      O método condiz com os objetivos e é adequado
      para um TCC?                                        & & & & & \\ \hline
    8. Foi definido um cronograma coerente com o método
      definido (indicando todas as atividades) e com as
      datas das entregas (p.ex. Projeto I, II, Defesa)?   & & & & & \\ \hline
    9. Foram identificados custos relativos à execução
      deste trabalho (se houver)? Haverá financiamento
      para estes custos?                                  & & & & & \\ \hline
    10. Foram identificados todos os envolvidos neste
      trabalho?                                           & & & & & \\ \hline
    11. As formas de comunicação foram definidas
      (ex.: horários para orientação)?                    & & & & & \\ \hline
    12. Riscos potenciais que podem causar desvios do
      plano foram identificados?                          & & & & & \\ \hline
    13. Caso o TCC envolva a produção de um software ou
      outro tipo de produto e seja desenvolvido também
      como uma atividade realizada numa empresa ou
      laboratório, consta da proposta uma declaração
      (Anexo 3) de ciência e concordância com a entrega
      do código fonte e/ou documentação produzidos?       & & & & & \\ \hline
  \end{tabular}

  \vspace{2mm}
  {\footnotesize
  \begin{tabular}{|>{\bfseries}p{3cm}|l|l|l|}
    \hline Avaliação & \multicolumn{2}{l}{\bf $\square$ Aprovado}
      & \textbf{$\square$ Não Aprovado} \\
    \hline Professor Responsável & Ricardo Felipe Custódio & 12/06/2017 & \\
    \hline
  \end{tabular}}
\end{table}

\paginaresumo

\sumario

\chapter{Introdução}

Funções de resumo criptográficas são essenciais em diversas aplicações de
segurança da informação, como códigos de autenticação de mensagens (MACs),
identificação de arquivos a partir de uma `'impressão digital`' única, detecção
de perda de informações em uma transmissão volátil etc., e têm uso disseminado
em esquemas de assinatura digital, os quais são o foco deste trabalho. Uma
característica relevante destes é a possibilidade da demonstração matemática da
autenticidade de uma mensagem transmitida entre entidades, sejam estas
confiáveis ou não, utilizando as funções citadas anteriormente, para que seja
estabelecida uma comunicação segura.

A maior parte dos esquemas de assinatura digital, na prática, utilizam
criptossistemas de chave pública baseados em problemas de teoria de números --
fatoração de inteiros, logaritmo discreto -- que, atualmente, não podem ser
computados em tempo polinomial (ou seja, seu tempo de execução é limitado por
uma expressão polinomial relacionada com o tamanho da entrada do algoritmo).
Entretanto, utilizando-se de um computador quântico, tais problemas podem ser
resolvidos em tempo polinomial \cite{shor1999polynomial}.

A criptografia \emph{pós-quântica} encarrega-se de buscar algoritmos
criptográficos cuja complexidade independe de problemas futuramente
solucionáveis em teoria de números. Como visto em \cite{Merkle1990}, um
criptossistema de chaves públicas pode ser definido de forma independente da
função de resumo criptográfica utilizada, assim possibilitando a exploração de
diversas combinações, cujos elementos essenciais (e.g. árvores de Merkle) não
influenciam na segurança do esquema.

Neste trabalho, utilizamos funções de resumo criptográficas para a construção
de esquemas de assinatura digital, considerados seguros se e somente as funções
forem resistentes à colisão (a inviabilidade computacional de encontrar duas
mensagens distintas que, submetidas à mesma função, retornam a mesma saída),
tornando os requisitos de segurança mínimos para os esquemas.

\section{Objetivos}

\noindent \emph{Objetivo geral.} Apresentar um estudo detalhado sobre esquemas
de assinatura digital baseados em funções de resumo criptográficas, partindo de
esquemas de assinatura única \cite{Lamport1979}, observando o refinamento
destes, até o estado da arte, onde não é necessário saber quantas assinaturas
foram geradas anteriormente \cite{Bernstein2015}, bem como implementações em
linguagem de alto nível para a fácil compreensão destes esquemas. \\

\noindent \emph{Objetivos específicos.} Descrever os esquemas de assinatura
digital única Lamport-Diffie e Winternitz; descrever os esquemas de assinatura
digital baseado em árvores de Merkle -- \emph{Merkle Signature Scheme},
\emph{eXtended Merkle Signature Scheme}; implementar os esquemas supracitados;
comparar o desempenho destes algoritmos entre si, utilizando funções de resumo
criptográficas e parâmetros internos aos algoritmos distintos, onde aplicável.
\\

\noindent \emph{Escopo do trabalho.} Não se aplica ao conteúdo deste trabalho
a análise profunda de provas de segurança definidas por um modelo adversarial
teórico -- ou seja, demonstrar que um atacante deve resolver um problema muito
difícil para tornar o algoritmo inseguro, bem como algoritmos de criptografia
pós-quânticos baseados em outras estruturas matemáticas (reticulados, teoria
de códigos etc.) ou algoritmos clássicos (RSA, DSA, ECDSA etc.). \\

\noindent \emph{Critérios de aceitação.} Estudo e implementação de pelo menos
dois esquemas de assinatura digital única (Lamport-Diffie, Winternitz),
a estrutura de dados chamada de árvore de Merkle e um esquema de assinatura
digital composto da união destes elementos, como o XMSS \cite{Buchmann2011}. \\

\noindent \emph{Entregas do projeto.} Relatórios referentes às disciplinas de
Trabalho de Conclusão de Curso do INE/UFSC, incluindo monografia completa ao
final da disciplina de Trabalho de Conclusão de Curso II, bem como
implementação documentada em linguagem de programação de alto nível dos
esquemas descritos no trabalho. \\

\noindent \emph{Restrições e premissas.} Espera-se reunir com os orientadores,
de forma periódica, para a discussão dos resultados obtidos e a definição dos
passos seguintes. As restrições consistem na finalização do projeto até o prazo
de entrega final da disciplina de Trabalho de Conclusão de Curso II do
INE/UFSC, utilização de software livre e de código aberto, normatização dos
documentos referentes ao projeto de acordo com órgãos especializados (ABNT,
BU/UFSC) e disponibilização de artigos e livros pagos e/ou gratuitos referentes
aos esquemas abordados.

\section{Procedimentos metodológicos}

O trabalho será desenvolvido utilizando a infraestrutura e recursos do
Laboratório de Segurança em Computação (LabSEC/UFSC), onde será estudada
bibliografia referente aos temas abordados nesta pesquisa buscando encontrar
as vantagens e desvantagens entre cada um dos esquemas de assinatura digital
escolhidos, bem como observar seu desempenho e tamanho de elementos como
par de chaves e assinatura, ao utilizar funções de resumo criptográficas
distintas em implementações produzidas ou fornecidas.

\chapter{Cronograma}

\begin{figure}[htbp]
  \begin{tabular}{|p{4.04cm}|*{6}{c|}}
    \hline \rowcolor{lightgray}
      & \multicolumn{6}{c|}{\textbf{Meses -- 2017}} \\
    \hhline{|>{\arrayrulecolor{lightgray}}->{\arrayrulecolor{black}}|
      |------>{\arrayrulecolor{lightgray}}>{\arrayrulecolor{black}}|}
    \rowcolor{lightgray}
      \multicolumn{1}{|c|}{\multirow{-2}{*}{\textbf{Etapas}}}
      & \textbf{jul.} & \textbf{ago.} & \textbf{set.}
      & \textbf{out.} & \textbf{nov.} & \textbf{dez.} \\
    \hline Fundamentação teórica & \cellcolor{lightgray} & & & & & \\
    \hline Revisão do estado da arte & \cellcolor{lightgray}
      & \cellcolor{lightgray} & & & & \\
      \hline Desenvolvimento do TCC & & \cellcolor{lightgray}
      & \cellcolor{lightgray} & \cellcolor{lightgray} & & \\
    \hline Implementação & & & & \cellcolor{lightgray}
      & \cellcolor{lightgray} & \cellcolor{lightgray} \\
    \hline Relatório de TCC I & & & & & \cellcolor{lightgray} & \\
    \hline
    \hline \rowcolor{lightgray}
      & \multicolumn{6}{c|}{\textbf{Meses -- 2018}} \\
    \hhline{|>{\arrayrulecolor{lightgray}}->{\arrayrulecolor{black}}|
      |------>{\arrayrulecolor{lightgray}}>{\arrayrulecolor{black}}|}
    \rowcolor{lightgray}
      \multicolumn{1}{|c|}{\multirow{-2}{*}{\textbf{Etapas}}}
      & \textbf{jan.} & \textbf{fev.} & \textbf{mar.}
      & \textbf{abr.} & \textbf{mai.} & \textbf{jun.} \\
    \hline Ajustes na implementação & \cellcolor{lightgray} & & & & & \\
    \hline Redação da monografia & \cellcolor{lightgray}
      & \cellcolor{lightgray} & \cellcolor{lightgray} & & & \\
    \hline Ajustes na monografia & & & \cellcolor{lightgray}
      & \cellcolor{lightgray} & & \\
    \hline Relatório de TCC II & & & & & \cellcolor{lightgray} & \\
    \hline Defesa pública & & & & & & \cellcolor{lightgray} \\
    \hline Ajustes finais do TCC & & & & & & \cellcolor{lightgray} \\
    \hline
  \end{tabular}
\end{figure}

\chapter{Custos}

\begin{figure}[htbp]
  \begin{tabular}{|p{1.69cm}|*{3}{l|}}
    \hline \thatcell{1}{|c|}{Item} & \thatcell{1}{c|}{Quantidade}
      & \thatcell{1}{c|}{Valor unitário (R\$)}
      & \thatcell{1}{c|}{Valor Total (R\$)}                         \\
    \hline \thatcell{4}{|l|}{Material permanente}                   \\
    \hline Computador   & 1     & R\$ 3.000,00  & R\$ 3.000,00      \\
    \hline Internet     & 1     & R\$ 1.000,00  & R\$ 1.000,00      \\
    \hline Artigos      & 10    & R\$ 90,00     & R\$ 900,00        \\
    \hline Livros       & 2     & R\$ 200,00    & R\$ 400,00        \\
    \hline \thatcell{4}{|l|}{Material de consumo}                   \\
    \hline Alimentação  & 264   & R\$ 10,00     & R\$ 2.640,00      \\
    \hline CDs/DVDs     & 4     & R\$ 2,00      & R\$ 8,00          \\
    \hline \thatcell{4}{|l|}{Outros recursos e serviços}            \\
    \hline Impressões   & 200   & R\$ 1,00      & R\$ 200,00        \\
    \hline
  \end{tabular}
\end{figure}

\chapter{Recursos Humanos}

\begin{figure}[htbp]
  \begin{tabular}{|*{2}{p{4.96cm}|}}
    \hline \rowcolor{lightgray}
    \thatcell{1}{|c|}{Nome}       & \thatcell{1}{c|}{Função}    \\
    \hline Gustavo Zambonin       & Autor                       \\
    \hline Ricardo F. Custódio    & Orientador                  \\
    \hline Daniel Panario         & Coorientador                \\
    \hline Renato Cislaghi        & Coordenador de projetos     \\
    \hline A definir              & Membro(s) da banca          \\
    \hline
  \end{tabular}
\end{figure}

\chapter{Comunicação}

\begin{figure}[htbp]
  \footnotesize
  \begin{tabular}{|P{1.9cm}|P{0.7cm}|P{2.5cm}|P{1.5cm}|P{2cm}|}
    \hline \rowcolor{lightgray}
    \textbf{O que precisa ser comunicado} & \textbf{Por quem}
      & \textbf{Para quem} & \textbf{Melhor forma de comunicação}
      & \textbf{Quando e com que frequência} \\
    \hline Enviar plano de projeto & Autor
      & Orientador, coorientador, coordenador de projetos & Sistema de TCC
      & Única vez, até dia 12/06/2017 \\
    \hline Entrega de relatório de TCC I & Autor
      & Orientador, coorientador, coordenador de projetos,
      membro(s) da banca & Sistema de TCC
      & Única vez, ao final do semestre 2017/2 \\
    \hline Entrega de relatório de TCC II & Autor
      & Orientador, coorientador, coordenador de projetos,
      membros(s) da banca & Sistema de TCC
      & Única vez, em meados do semestre 2018/1 \\
    \hline Defesa do TCC & Autor
      & Orientador, coorientador, coordenador de projetos,
      membro(s) da banca & Pessoalmente
      & Única vez, em meados do semestre 2018/1 \\
    \hline Entrega final da monografia & Autor
      & Orientador, coorientador, coordenador de projetos,
      membro(s) da banca & Sistema de TCC
      & Única vez, após a defesa, antes do término de 2018/1 \\
    \hline Reuniões de acompanhamento da pesquisa & Autor
      & Orientador, coorientador & Pessoalmente, webconferência
      & Quinzenalmente \\
    \hline Monitoramento do projeto & Autor
      & Orientador, coorientador & E-mail & Esporadicamente \\
    \hline Convite de membro(s) da banca & Autor & A definir
      & Sistema de TCC & Única vez, em meados do semestre 2017/2 \\
    \hline
  \end{tabular}
\end{figure}

\chapter{Riscos}

\begin{figure}[htbp]
  \footnotesize
    \begin{tabular}{|P{1.3cm}|*{3}{P{0.8cm}|}*{2}{P{2.24cm}|}}
    \hline \rowcolor{lightgray}
    \textbf{Risco} & \textbf{Proba\-bilidade} & \textbf{Impacto}
      & \textbf{Priori\-dade} & \textbf{Estratégia de resposta}
      & \textbf{Ações de prevenção} \\
    \hline Paralisação de transporte público & Média & Médio & Baixa
      & Transportar-se à Universidade utilizando meios alternativos
      & Combinar transporte alternativo \\
    \hline Paralisação de servidores públicos & Muito baixa & Alto & Média
      & Produzir monografia e pesquisa utilizando recursos pessoais
      & Não se aplica \\
    \hline Problemas de saúde & Baixa & Alto & Alta
      & Tratamento médico das condições identificadas
      & Diminuição do fator de exposição em caso de doenças com fator
        ambiental, e exames para verificar condições genéticas \\
    \hline Perda de dados & Muito baixa & Alto & Média
      & Recuperar cópia de segurança
      & Cópias de segurança periódicas do material produzido \\
    \hline Queima de equipamento(s) eletrônico(s) & Muito baixa & Alto & Média
      & Comprar novo(s) equipamento(s)
      & Evitar utilização do(s) equipamento(s) em más condições de tempo ou
        por períodos muito prolongados \\
    \hline
  \end{tabular}
\end{figure}

\bibliographystyle{abnt-alf}
\bibliography{\jobname}

\end{document}
